\documentclass{article}
\usepackage[utf8]{inputenc}
\usepackage{amsmath}
\usepackage{amssymb}

\title{Lightening Presentation Speech}
\author{Junliang Zhou}

\begin{document}
\maketitle

Hello everyone. I am Jimmy and this is my partner Likun. \newline

Today we are presenting an enhanced LLOB model with the existence of multi-timescale market agents developed by Benzaquen and Bouchaud. The model solves several deficiencies of the original LLOB model to make it more coincide with real market data. \newline

I believe everyone here is quite familiar with LLOB model. So I'll only make a quick introduction. The LLOB model, or locally linear order book model, describes the density of limit orders in a latent order book. The densities of bid and ask are given by the following PDEs, while the price are the level where the two densities are equal. \newline

We can easily solve the stationary order book at equilibrium, which is Equation 3. And the shape of order book is shown in Figure 1. From the figure we know that there is a linear part in the order book and order quantity of this part is $Q_\text{lin}$. \newline

Although LLOB model successfully provided a theoretical approach to modeling the square-root impact law. There are several drawbacks of it. \newline

Firstly, the original LLOB model focus only on the infinite memory limit, where the cancellation and deposition rate both go to 0. \newline

Secondly, the square-root law is only recovered when execution rate of meta-orders is extremely large, even larger than the execution rate of market itself. \newline

Thirdly, it introduces a strong mean-reversion effect which is not observed in real prices. This is referred to the "diffusivity puzzle". \newline

Now we tackle the problems one by one. \newline

When the order book has finite memory, where cancellation and deposition rates are non-zero, there are multiple situations to consider. Whether the participation rate of meta-order is small or large. Whether the execution is fast or slow. Whether the meta-order volume is large or small. \newline

Market impact has different profiles under different conditions. When the execution is fast and volume is small, we can recover a square-root impact with the cancellation rate as a variable. While this is a linear market impact in the opposite situation. \newline

From Figure 2 we know that the second term of Equation 5 offsets some of the market impact. The reasons lead to the different results are actually the same. As the cancellation and deposition rate increases comparing to the execution rate, the order book essentially renews itself faster and the information brought by the meta-order will be lost, or covered, by the renewal. \newline

As part of the information is lost during or after the execution of meta-order, it is not a surprise that a permanent impact would be found in the price trajectory. And we showed that it's linear to the meta-order size. \newline

Now we are going to see how the square-root market impact is achieved when the meta-order participation rate is much smaller than the market by introducing a double-frequency framework. \newline

Consider there are two sorts of agents co-exists in the market. One is slow agent like normal investors. The other is fast agent like HFTs and market makers with large cancellation and deposition rates. Clearly the market turnover is dominated by the latter one. \newline

We can solve the order book shape for each kind of agents. The slow one is asymptotically linear function, like the agent in the infinite memory LLOB model. The fast one is asymptotically a stepwise function. Therefore the total order book shape is the red/blue line shown in Figure 3. \newline

Let's consider the meta-order intensity is in between of the slow and fast agents. We can solve the meta-order participation rate for both agents and the price trajectory. \newline

The result shows that the incoming meta-order is executed mainly by rapid agents at the beginning, but the slow agents will take over gradually. From the different nature of order books of two types of agents, that one is stepwise and the other is linear, the market impact is linear when the fast agents are dominating at the beginning, and then becomes square-root when the slow agents are taking over. \newline

Therefore we successfully recovered the square-root law for a reasonable scale of meta-order intensity that is compatible with real market. \newline

We can furthermore extend the model to a multi-frequency framework, where there is a continuous range of cancellation and deposition rates. The setup is very similar to the double-frequency one. The only difference is the introduction of $\rho(\nu)$, which is the distribution of cancellation rate. \newline

Here we mainly focus on the resolution of the "diffusivity puzzle". The price diffusion in original LLOB model converges as time goes, which leads to a strong mean-reverting effect. By introducing a fat-tailed distribution, namely power-law distribution, to the cancellation rate, the price diffusion becomes Equation 16 and it's divergent. \newline

Therefore, the mean-reversion effects induced by a persistent order book can exactly offset trending effects induced by a persistent order flow. And then resolved the "diffusivity puzzle". \newline

Also we can see from Figure 5 that the price trajectory in the multi-timescale framework is quite similar to it in the double-frequency. The market impact follows a power law with order of 3/4 and 5/8 respectively. \newline

To sum up, the authors modified the origin LLOB by introducing multi-timescale agents to make the model consistent with the real market data, especially with square-root market impact and diffusive price. \newline

Thank you for your attention! And if you have any questions, please don't hesitate to ask. \newline

\end{document}
